\section{Обзор существующих подходов}
Цель данного обзора состоит в исследовании основных подходов к решению рассматриваемой задачи, изучении существующих моделей представления сложноструктурированных событий и методов, применяемых для прогнозирования потоков таких событий.

\subsection{Представление событий}
% (общее, теория) \\
Первым этапом исследований во всех рассмотренных работах был переход от потоков событий к временным рядам.

\subsubsection{Простые события}\label{no_struct_events}
Переход к временным рядам не составляет сложности в задачах, где у событий признаки отсутствуют и ставится задача предсказания количества событий, которые произойдут в течение некоторого временного периода. В таком случае какое-то определенное представление событий не требуется и поток событий просто преобразуется в одномерный временной ряд, где каждое значение это количество произошедших событий. Далее в этих работах авторы сразу переходят к предложениям по прогнозированию таких временных рядов.

Таким образом и поступают авторы \cite{traffic_flow_forecasting}, при решении задачи прогнозирования автомобильного траффика. Хотя в этой статье и описывается подход прогнозирования траффика с использованием разнородных источников данных, эти данные представляют из себя одномерные временные ряды, что избавляет авторов от задачи поиска подходящей модели представления событий. В работе \cite{modeling_patterns_with_DNN} и \cite{predictive_monitoring_LSTM} авторы акцентируют внимание на решении задачи прогнозирования и поиска паттернов во временных рядах, опять же, без необходимости реализации какого-то сложного представления данных.

\subsubsection{События с относительно простой структурой} \label{simple_struct_events}
В задачах, где событие описывается какими-то признаками, подходы к представлению таких событий могут быть сильно различаться от задачи к задаче, но многие из них объединяет общий подход, состоящий в формировании формулы для вычисления значений признаков для событий экспертами, следовательно эти признаки специфичны для какой-то отдельно взятой области, и их не получится легко перенести на задачу из другой области.

Так, например, в работе \cite{twitter_predicting} событиями являются посты в сети Twitter (твиты) и для этих событий создаются различные ориентированные на их структуру (текст, возможно ссылки, хэштеги) признаки, а так же признаки, специфичные для задачи, такие как исторические изменения популярности каких-то тем, соответствующих твиту. 
В работе \cite{triplet_event_forecasting}, каждое событие представляет собой триплет вида (объект -- URL, субъект -- пользователь, временная метка -- время клика), а в качестве признаков для прогнозирования используются взаимодействия объектов и субъектов на протяжении какого-то временного интервала.

В работе \cite{uber_mulitvar_forecasting} описание событий является многомерным временным рядом, такое представление является более сложным, чем представление, описанное в \ref{no_struct_events}, но оно так же поддерживает возможность агрегации по временным периодам (например, усреднением) и не требует какой-то предобработки признаков событий.

\subsubsection{События с разнородными признаками}
В задачах, где события описываются одним и тем же набором числовых, категориальных, бинарных и, возможно, текстовых признаков, необходимо использовать какое-то представление для событий.

С бинарными признаками в общем случае можно работать как с числовыми, а категориальные признаки приводятся к бинарным с помощью подхода One-Hot Encoding или обрабатываются любым другим способом (value encoding/label encoding/target encoding)
С текстовыми описаниями все сложнее, но в общем виде все равно требуется получить какое-то числовое (векторное) представление текста. Для решения этой задачи можно применять как классические подходы к тематическому моделированию, (TF-IDF, LDA), так и более современные подходы, в том числе использование word2vec или LSTM для получения эбмеддингов (векторных представлений) слов, предложений или целых документов.
После обработки, все эти признаки можно объединить в новый вектор признаков, которые будут исключительно числовыми, и дальше работать с новыми признаками как описано в \ref{simple_struct_events}

В случае со сложноструктурированными событиями иногда часть признаков не используется, так в работе \cite{comparison_coronary_heart_disease} из 16 разнородных медицинских признаков были выделены 10 числовых признаков и прогнозирование заболеваний сердца производилось только с их использованием. Такой подход имеет право на существование, но не использует всю доступную информацию.

В работе \cite{struct_unstruct_prediction} используются табличные данные, которые по своей природе очень просто привести к временным рядам, вместе с ними используются неструктурированные текстовые данные, полученные из открытых источников, затем производят многоклассовую классификацию каждого документа, где классы это набор заранее определенных рубрик. После классификации авторы получают из неструктурированных текстовых данных многомерный числовой ряд (вероятности принадлежности текстового документа к конкретной рубрике), размерность которого соответствует количеству различных рубрик. В итоге авторы используют этот временной ряд наряду со структурированными данными для прогнозирования.

\subsection{Прогнозирование событий}
% (общее, теория) \\
Задачей прогнозирования событий в общем случае является нахождение значений целевой переменной отклика в какой-то момент времени в будущем. Чаще всего рассматривается одна из подзадач: определение количества событий, которые произойдут, или определение вероятности того, что событие произойдет.

%%% REGRESSION
\subsubsection{Задача прогнозирования количества событий}
Особенностью задачи прогнозирования количества событий является то, что целевая переменная (отклик) может принимать произвольные (обычно неотрицательные) значения. В качестве примера можно привести задачу прогнозирования количества продаж товара в течении следующей недели, количества заказов такси в течение следующих суток и многие другие задачи, встречающиеся в реальной жизни. Такие задачи прогнозирования количества событий мы будем называть задачами регрессии.

Авторы многих работ предлагают решения, основанные на применении глубоких нейронных сетей, в частности авторы \cite{traffic_flow_forecasting} предлагают использовать связку CNN и RNN для прогнозирования дорожного траффика на каком-то определенном участке дороги в каком-то определенном временном интервале, то есть решается задача регрессии.

В работе \cite{modeling_patterns_with_DNN} авторы также используют сочетание CNN и RNN для прогнозирования многомерных временных рядов, т.е. для решения задачи регрессии значений временного ряда. Особенность сочетания CNN и RNN в том, что CNN позволяет моделировать локальные зависимости на коротких промежутках времени, а RNN позволяет моделировать долгосрочные временные зависимости.

Авторы \cite{predictive_monitoring_LSTM} используют LSTM для задачи прогнозирования логов, например, для предсказывания времени, которое потребуется на завершение некоторой задачи.

В работе \cite{uber_mulitvar_forecasting} решается чистая задача регрессии: авторы с помощью рекуррентных нейронных сетей прогнозируют количество вызовов такси в заданном временном интервале. 

Подход авторов работы \cite{triplet_event_forecasting} отличается от перечисленных выше. В этой работе авторы, решая общую задачу поиска паттернов в переходах пользователей по различным URL, в частности рассматривая задачу прогнозирования объема траффика от конкретного пользователя (задачу регрессии). Для ее решения применяются авторегрессионные методы.

%%% CLASSIFICATION
\subsubsection{Задача прогнозировании вероятности события}
Определение вероятности наступления события -- задача, в которой значение целевой переменной (отклика) характеризует вероятность наступления события. Чаще всего в таких задачах отклик принимает значения в диапазоне от 0 до 1. Задачи прогнозирования такого рода мы будем называть задачами классификации.

Задачи классификации охватывают широкий спектр проблем и включают в себя задачу бинарной классификации (напр. произойдет или нет определенное событие в будущем), задачу многоклассовой классификации (напр. какое событие из нескольких возможных с наибольшей вероятностью произойдет в будущем) и задачу поиска аномалий (напр. определение паттерна, которые предшествует некоторым аномальным событиям).

Авторы \cite{struct_unstruct_prediction} решают задачу классификации, прогнозируя направление движения цены акций (вверх или вниз) и используют для этого нейронные сети с архитектурой LSTM. В качестве признаков используются как структурированные исторические данные из открытых источников, так и неструктурированные текстовые данные, обработанные специальным образом.

В работе \cite{twitter_predicting} описан следующий подход -- авторы на основании экспертных знаний формируют признаки, и используя значения этих признаков и некоторых исторических данных обучают модель классификации для прогнозируют тренда популярности определенного события. Таким образом эта работа описывает задачу многоклассовой классификации с классами "популярность вырастет", "останется на том же уровне" и 3 класса, описывающих различные по силе падения популярности.

Работа \cite{comparison_coronary_heart_disease} напрямую не описывает задачу прогнозирования событий, в этой работе рассматривается вероятность того, что событие произойдет когда-либо в будущем, и поскольку четкой авторы не используют какую-то временн\textit{у}ю структуру для описания признаков или событий, то эта задача сводится к задаче бинарной классификации и уже для этой задачи авторы сравнивают различные существующие методы классификации.

Отдельно стоит рассматривать задачу прогнозирования экстремально редких событий. Поскольку в такой задаче набор данных имеет ярко выраженный дисбаланс классов, авторы в большинстве случаев переходят от решения задачи прогнозирования к задаче поиска аномалий, таким образом они не предсказывают события, но определяют характерные паттерны, предшествующие этим редким событиям.

Для решения таких задач в качестве стандартных средств часто используются одноклассовые классификаторы, например в работе \cite{anomaly_detection_oneclass} авторы с помощью обучают одноклассовый классификатор на показаниях с датчиков станка и выявляют паттерны, предшествующие выходу оборудования из строя.

Авторы \cite{anomaly_detection_ae} предлагают использовать более продвинутые средства -- автокодировщики -- для решения задачи прогнозирования выхода оборудования из строя. В этой работе авторы высказывают предположение, что если автокодировщик, обученный на наборе показаний датчиков в случае штатной работы оборудования показывает большую ошибку в задаче кодирования и последующего декодирования показаний датчиков, то такие показания можно считать аномальными.

\subsection{Выводы из обзора}
Было проведено исследование основных подходов к двум этапам решения рассматриваемой задачи, изучены существующие методы представления как сложноструктурированных, так и достаточно простых объектов и методы, применяемые для прогнозирования временных рядов. По результатам исследования сделаны следующие выводы:
\begin{itemize}
    \item для представления событий чаще всего используются признаки: описанные экспертами и соответствующие тематике задачи (\cite{twitter_predicting}), в некоторых работах ищется автоматическое представление, но только для текстовых описаний (\cite{struct_unstruct_prediction}).
    \item для прогнозирования событий чаще всего используются нейронные сети (CNN + RNN: \cite{traffic_flow_forecasting}, \cite{modeling_patterns_with_DNN}; LSTM: \cite{uber_mulitvar_forecasting}, \cite{predictive_monitoring_LSTM}, \cite{struct_unstruct_prediction});
    \item для экстремально редких событий можно использовать подход поиска аномалий, т.к. остальные методы прогнозирования показывают себя плохо при сильном дисбалансе классов.
    \item не существует какого-то общего подхода, объединяющего одновременно поиск представления сложноструктурированных событий, и прогнозирование этих событий.
\end{itemize}

Основным направлением дальнейшей работы будет являться как исследование и разработка представления сложноструктурированных событий в виде исключительно числовых признаков, так и прогнозирование этих событий.
