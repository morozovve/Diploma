\section{Заключение}
В данной работе рассматривались методы представления сложноструктурированных событий и прогнозирования их потоков. По результатам исследования были выявлены ограничения существующих методов решения задачи и предложен подход, позволяющий одновременно получать универсальное представление сложноструктурированных событий в виде временного ряда и производить их прогнозирование.
Ранее в работах со схожей тематикой не рассматривалось какое-либо автоматическое выделение признаков для дальнейшего прогнозирования. Подход, описанный в данной работе, объединяет автоматизацию выделения признаков, достаточно универсальных и поддающихся агрегации и интерпретации с построением модели прогнозирования событий по выделенным признакам.
Во время работы над построением представления сложноструктурированных событий был разработан дополнительный метод, позволяющий сопоставить ключевые слова автоматически выделенным тематикам, т.о. позволяющий строить описание события с помощью ключевых слов на основании значений тематик для данного события.
По итогам проведения экспериментов были найдены модели прогнозирования и способы построения представления сложноструктурированных событий, показывающие наилучшие результаты прогнозирования на тестовых данных.
Таким образом, главными итогами данной  работы являются следующие результаты:
\begin{enumerate}
    \item Предложено и реализовано представление данных из базы сложноструктурированных событий, основанное на методах понижения размерности и позволяющее работать со всеми типами признаков: текстовыми, числовыми, категориальными
    \item Признаки, полученные с использованием предложенного метода представления сложноструктурированных событий, могут быть использованы при решении задачи прогнозирования событий
    \item Разработан вспомогательный способ интерпретации скрытых признаков (тематик) через доступные текстовые описания событий, позволяющий интерпретировать даже поведение тех скрытых признаков, которые были получены сложными нелинейными преобразованиями исходных данных
    \item Проведены эксперименты и оценена работа различных моделей для решения задач прогнозирования как количества событий, так и вероятности наступления события

\end{enumerate}